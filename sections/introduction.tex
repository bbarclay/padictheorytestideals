\section{Introduction}

\subsection{Background and Motivation}

The theory of test ideals has played a central role in commutative algebra and algebraic geometry for the past three decades. Test ideals were originally introduced by Hochster and Huneke \cite{HH90} as a characteristic $p > 0$ tool to study tight closure, a closure operation that captures subtle aspects of rings in positive characteristic. In characteristic 0, multiplier ideals serve an analogous purpose and have become fundamental in birational geometry \cite{Laz04}. Both notions provide sophisticated ways to measure singularities of algebraic varieties, with applications ranging from the minimal model program to bounds on symbolic powers of ideals.

However, extending these theories to mixed characteristic settings—rings and schemes whose generic point has characteristic 0 but special points have positive characteristic—has presented significant challenges. The absence of the Frobenius endomorphism in characteristic 0 requires different techniques than those used in positive characteristic, while the arithmetic complications of mixed characteristic demand innovative approaches.

\subsection{Recent Developments}

Recent advancements in p-adic geometry, particularly the theory of perfectoid spaces developed by Scholze \cite{Sch12}, have provided new tools for tackling problems in mixed characteristic. Building on these developments, Ma and Schwede \cite{MS21} introduced a notion of test ideals in mixed characteristic using the "plus closure" operation. Further work by Takamatsu and Yoshikawa \cite{TY21} and Bhatt et al. \cite{BMP+23} has expanded our understanding of these objects.

Despite these advances, three fundamental problems have remained open and represent significant barriers to developing a complete theory:

\begin{enumerate}
    \item \textbf{Completion Problem:} In characteristic $p > 0$, test ideals are known to commute with completion under mild conditions. Does $\tau_+(R,\Delta)$ commute with completion in mixed characteristic? Specifically, is $\tau_+(\hat{R},\hat{\Delta}) \cap R = \tau_+(R,\Delta)$?
    
    \item \textbf{Subadditivity Problem:} A fundamental property of test ideals in characteristic $p > 0$ and multiplier ideals in characteristic 0 is subadditivity. For divisors $\Delta_1$ and $\Delta_2$, does the subadditivity property $\tau_+(R,\Delta_1+\Delta_2) \subseteq \tau_+(R,\Delta_1) \cdot \tau_+(R,\Delta_2)$ hold in mixed characteristic?
    
    \item \textbf{Alternative Formulations Problem:} Multiple formulations of test ideals have emerged in mixed characteristic (standard, trace-based, perfectoid, tight closure). How do these different formulations relate to each other? Is there a unified framework that explains when they agree and when they differ?
\end{enumerate}

\subsection{Our Contribution}

In this paper, I introduce a novel approach based on binary p-adic patterns that resolves all three problems simultaneously. The key insight is that test ideal membership in mixed characteristic can be precisely characterized by predicates on the p-adic digit representations of ring elements. This characterization transforms complex algebraic conditions into computational predicates that can be systematically analyzed.

Our main contribution is threefold:

\begin{enumerate}
    \item We develop a comprehensive binary p-adic framework that characterizes test ideal membership through explicit predicates on p-adic digit patterns.
    
    \item We prove that test ideals commute with completion in mixed characteristic, with the precise relationship governed by these binary predicates.
    
    \item We establish the subadditivity property for test ideals in mixed characteristic through a novel perfectoid factorization theory, which decomposes elements based on their binary p-adic patterns.
    
    \item We unify the various formulations of test ideals in mixed characteristic through a master binary predicate with specific modifications that precisely characterize when and how the formulations differ.
\end{enumerate}

This binary p-adic approach provides a powerful new paradigm for understanding test ideals in mixed characteristic, bridging the gap between characteristic $p > 0$ and characteristic 0 theories. By casting algebraic properties in terms of digit patterns, we obtain a computationally accessible framework that yields surprising insights into the structure of mixed characteristic rings.

\subsection{Organization of the Paper}

The remainder of this paper is organized as follows:

In Section \ref{sec:preliminaries}, we review the necessary background on p-adic expansions, test ideals in various characteristics, and perfectoid algebras.

Section \ref{sec:binary_framework} introduces the binary p-adic framework, developing the theory of binary predicates that characterize test ideal membership.

Section \ref{sec:completion_theorem} addresses the completion problem, proving that test ideals commute with completion through precise binary predicates.

In Section \ref{sec:subadditivity}, we establish the subadditivity property for test ideals in mixed characteristic using perfectoid factorization theory.

Section \ref{sec:alternative_formulations} unifies the various formulations of test ideals through modifications of the master binary predicate.

Section \ref{sec:global_properties} verifies that our approach satisfies all necessary schema-theoretic properties for a global theory.

Section \ref{sec:applications} presents applications and examples illustrating the computational power of the binary p-adic approach.

Finally, Section \ref{sec:conclusion} summarizes our results and discusses directions for future research. 