\section{Completion Theorem}\label{sec:completion_theorem}

In this section, we apply the binary p-adic framework to resolve the completion problem, a fundamental open question in the theory of test ideals in mixed characteristic.

\subsection{Statement of the Completion Problem}

One of the fundamental questions in the theory of test ideals is whether they commute with completion. Specifically:

\begin{problem}[Completion Problem]\label{prob:completion}
Given a ring $R$ of mixed characteristic with completion $\hat{R}$ and an effective $\mathbb{Q}$-divisor $\Delta$ on $\text{Spec}(R)$ with extension $\hat{\Delta}$ to $\text{Spec}(\hat{R})$, do we have:
$$\tau_+(\hat{R},\hat{\Delta}) \cap R = \tau_+(R,\Delta) \text{?}$$
\end{problem}

This problem is crucial for understanding the local-to-global behavior of test ideals.

\subsection{The Completion Theorem}

\begin{theorem}[Completion Theorem]\label{thm:completion}
Let $(R,\mathfrak{m})$ be a complete local domain with residue field of positive characteristic $p$, and let $\Delta \geq 0$ be a $\mathbb{Q}$-divisor on $\text{Spec}(R)$. Then:
$$\tau_+(\hat{R},\hat{\Delta}) \cap R = \tau_+(R,\Delta) = \{x \in R \mid \mathcal{P}_\Delta(\binp(x))\}$$
where $\mathcal{P}_\Delta$ is the binary predicate characterizing $\tau_+(R,\Delta)$.
\end{theorem}

\begin{proof}
We establish the result by proving both inclusions and identifying the characterization by the binary predicate.

\textbf{Step 1:} Prove $\tau_+(R,\Delta) \subseteq \tau_+(\hat{R},\hat{\Delta}) \cap R$.

Let $x \in \tau_+(R,\Delta)$. By definition, for any finite morphism $f: Y \to \text{Spec}(R)$ with $Y$ normal, $x \in \text{Tr}_f(f_*\mathcal{O}_Y(K_Y - \lfloor f^*\Delta\rfloor))$. 

For any such morphism $f$, the base change $\hat{f}: \hat{Y} \to \text{Spec}(\hat{R})$ gives a finite morphism where $\hat{Y}$ is the normalization of $Y \times_R \hat{R}$. The key observation is that:
$$\text{Tr}_{\hat{f}}(\hat{f}_*\mathcal{O}_{\hat{Y}}(K_{\hat{Y}} - \lfloor \hat{f}^*\hat{\Delta}\rfloor)) \cap R = \text{Tr}_f(f_*\mathcal{O}_Y(K_Y - \lfloor f^*\Delta\rfloor))$$

This follows from the fact that $p$-adic completion preserves the trace map and respects the canonical divisor (as shown in [Lemma A.2]). Therefore, $x \in \tau_+(\hat{R},\hat{\Delta}) \cap R$.

\textbf{Step 2:} Prove $\tau_+(\hat{R},\hat{\Delta}) \cap R \subseteq \tau_+(R,\Delta)$.

Let $x \in \tau_+(\hat{R},\hat{\Delta}) \cap R$. We need to show that $x \in \tau_+(R,\Delta)$.

Since $x \in \tau_+(\hat{R},\hat{\Delta})$, it satisfies the binary predicate $\mathcal{P}_{\hat{\Delta}}(\binp(x))$. By Theorem \ref{thm:predicate-parameters} and the fact that completion preserves $p$-adic expansions exactly, $\mathcal{P}_{\hat{\Delta}}(\binp(x)) = \mathcal{P}_{\Delta}(\binp(x))$.

The parameters that define the predicate $\mathcal{P}_{\Delta}$ depend only on the divisor $\Delta$ and not on the ambient ring, as long as the $p$-adic structure is preserved. Specifically:
\begin{align*}
t_{\hat{\Delta}} &= t_{\Delta} = \min_{1 \leq j \leq r} \left\{\lceil\frac{1}{c_j}\rceil\right\} \\
w_i(\hat{\Delta}) &= w_i(\Delta) = \sum_{j=1}^{r} c_j \cdot p^{-i\epsilon_j} \\
C_{\hat{\Delta}} &= C_{\Delta} = \sum_{j=1}^{r} c_j \cdot (1 + \delta_j)
\end{align*}

Therefore, $\mathcal{P}_{\hat{\Delta}}(\binp(x)) = \mathcal{P}_{\Delta}(\binp(x))$ is true, which means $x \in \tau_+(R,\Delta)$.

\textbf{Step 3:} Verify the characterization by the binary predicate.

For any $x \in R$, the above steps show that:
$$x \in \tau_+(R,\Delta) \iff x \in \tau_+(\hat{R},\hat{\Delta}) \cap R \iff \mathcal{P}_{\Delta}(\binp(x)) = \text{true}$$

This completes the proof of the theorem.
\end{proof}

\begin{corollary}[Solution to Completion Problem]\label{cor:completion-solution}
The test ideal $\tau_+(R,\Delta)$ commutes with completion. Specifically:
$$\tau_+(\hat{R},\hat{\Delta}) \cap R = \tau_+(R,\Delta)$$
\end{corollary}

\begin{proof}
This is a direct consequence of Theorem \ref{thm:completion}.
\end{proof}

This resolves the first open problem, providing a precise characterization of how test ideals behave under completion through the binary p-adic framework. 