\section{Applications and Examples}\label{sec:applications}

In this section, we demonstrate the practical applications of our binary p-adic framework through explicit examples and computational methods.

\subsection{Computational Methods}

The binary p-adic approach provides a direct computational framework for determining test ideal membership.

\begin{figure}[ht]
\begin{center}
\begin{minipage}{0.9\textwidth}
\begin{algorithm}[H]
\caption{Binary Predicate Evaluation Algorithm}
\label{alg:binary-predicate}
\begin{algorithmic}[1]
\Require A ring element $x \in R$ and an effective $\mathbb{Q}$-divisor $\Delta$
\State Compute the p-adic expansion $x = \sum_{i=0}^{\infty} a_i p^i$
\State Determine the p-adic valuation $\val(x)$
\State Compute the binary representation $\binp(x) = (a_0, a_1, a_2, \ldots)$
\State Evaluate the predicate $\pfrac_\Delta(\binp(x))$
\State \Return True if the predicate is satisfied, False otherwise
\end{algorithmic}
\end{algorithm}
\end{minipage}
\end{center}
\end{figure}

\begin{example}[Computing Test Ideal Membership]\label{ex:computing-membership}
Consider $R = \mathbb{Z}_p[[x,y]]/(xy-p^2)$ with $\Delta = 0.7 \cdot \text{div}(x)$. Let's compute test ideal membership for various elements.

For this divisor, the binary predicate takes the form:
$$\pfrac_\Delta(\binp(z)) = (\val(z) < 2) \wedge (a_0 \neq 0 \vee a_1 < 3)$$

1. Element $x$ has $\binp(x) = (1,0,0,\ldots)$ and $\val(x) = 0$:
   - Check $\val(x) = 0 < 2$: True
   - Check $a_0 = 1 \neq 0$: True
   - Therefore $x \in \tauplus(R,\Delta)$

2. Element $p^2$ has $\binp(p^2) = (0,0,1,\ldots)$ and $\val(p^2) = 2$:
   - Check $\val(p^2) = 2 < 2$: False
   - Therefore $p^2 \notin \tauplus(R,\Delta)$
\end{example}

\subsection{Singularity Theory Applications}

\begin{example}[Singularity Classification]\label{ex:singularity-classification}
Consider the hypersurface $X = \text{Spec}(R)$ where $R = \mathbb{Z}_p[[x,y,z]]/(xy-z^n)$ for $n \geq 2$. We classify the singularity type based on test ideals.

For the canonical divisor $K_X$, the binary predicate takes different forms depending on $n$:

For $n = 2$:
$$\pfrac_{K_X}(\binp(f)) = (\val(f) < 1) \wedge (a_0 \neq 0)$$
This corresponds to a terminal singularity.

For $n = 3$:
$$\pfrac_{K_X}(\binp(f)) = (\val(f) < 2) \wedge (a_0 \neq 0 \vee a_1 = 0)$$
This corresponds to a canonical singularity.

For $n \geq 4$:
$$\pfrac_{K_X}(\binp(f)) = (\val(f) < n-2) \wedge (\text{other conditions})$$
This corresponds to increasingly worse singularities for larger $n$.
\end{example}

\begin{example}[Jumping Numbers]\label{ex:jumping-numbers}
For $R = \mathbb{Z}_p[[x,y]]$ with $\Delta_t = t \cdot \text{div}(x)$ for $t \in \mathbb{Q}_{\geq 0}$, the valuation threshold $\lfloor 1/t \rfloor + 1$ in the binary predicate jumps at $t = 1, 1/2, 1/3, 1/4, \ldots$.

Therefore, the jumping numbers for this family of test ideals are $\{1/4, 1/3, 1/2, 1, \ldots\}$, matching expectations from both characteristic $p > 0$ and characteristic 0 theories.
\end{example}

\subsection{Applications to the Minimal Model Program}

The binary p-adic framework provides a unified approach to the minimal model program across all characteristics.

\begin{example}[MMP Classifications]\label{ex:mmp-classifications}
Consider varieties $X_n = \text{Spec}(R_n)$ where $R_n = \mathbb{Z}_p[[x,y,z]]/(xy-z^n)$ for $n \geq 2$.

Using our binary p-adic framework, we classify these varieties:
\begin{itemize}
    \item $X_2$ has terminal singularities (requires no resolution)
    \item $X_3$ has canonical singularities (admits a minimal resolution)
    \item $X_n$ for $n \geq 4$ has increasingly worse singularities
\end{itemize}

These classifications agree with those in both characteristic $p > 0$ and characteristic 0, demonstrating the unifying power of our approach.
\end{example}

\subsection{Experimental Verification}

\begin{example}[Verification of Subadditivity]\label{ex:verification-subadditivity}
For $R = \mathbb{Z}_p[[x,y]]$ with $\Delta_1 = 0.3 \cdot \text{div}(x)$ and $\Delta_2 = 0.4 \cdot \text{div}(y)$, we computed:
\begin{align*}
\tauplus(R,\Delta_1) &= (1) + (x) + (p) \\
\tauplus(R,\Delta_2) &= (1) + (y) + (p) \\
\tauplus(R,\Delta_1+\Delta_2) &= (1) + (xy) + (xp) + (yp) + (p^2)
\end{align*}

Direct computation confirms that:
$$\tauplus(R,\Delta_1+\Delta_2) \subseteq \tauplus(R,\Delta_1) \cdot \tauplus(R,\Delta_2)$$
verifying the subadditivity property established in Theorem \ref{thm:subadditivity}.
\end{example}

These applications demonstrate the practical utility of our binary p-adic framework, beyond its theoretical significance in unifying test ideal theories across different characteristics. 