\section{Global Scheme-Theoretic Properties}\label{sec:global_properties}

In this section, we verify that the binary p-adic approach to test ideals extends properly to global schemes and satisfies all necessary schema-theoretic properties for a coherent theory.

\subsection{Global Theory of Test Ideals}

We begin by defining test ideals globally and verifying their coherence:

\begin{definition}[Global Test Ideal Sheaf]\label{def:global-test-ideal}
Let $X$ be a scheme of mixed characteristic with an effective $\mathbb{Q}$-divisor $\Delta$. The sheaf of test ideals $\tau_+(X,\Delta)$ is defined by:
$$\tau_+(X,\Delta)(U) = \tau_+(O_X(U),\Delta|_U)$$
for any open subset $U \subseteq X$.
\end{definition}

This definition extends the local notion of test ideals to the global setting, but we must verify that it produces a coherent sheaf.

\begin{theorem}[Global Coherence]\label{thm:global-coherence}
The binary p-adic approach produces a coherent sheaf of test ideals on any scheme $X$ of mixed characteristic.
\end{theorem}

\begin{proof}
We verify the three key conditions for sheaf coherence:

1. \textbf{Restriction Maps Consistency:} When restricting from an open set $U$ to a smaller open set $V$, the binary predicate $\mathcal{P}_{\Delta|_U}$ restricts to $\mathcal{P}_{\Delta|_V}$. This is because the p-adic structure of elements is preserved under restriction.

2. \textbf{Gluing Conditions:} For open subsets $U_i$ covering $U$, if a section $s \in O_X(U)$ has $s|_{U_i} \in \tau_+(O_X(U_i),\Delta|_{U_i})$ for all $i$, then $s \in \tau_+(O_X(U),\Delta|_U)$. This follows because:
   - The binary pattern $\binp(s|_{U_i})$ satisfies the predicates $\mathcal{P}_{\Delta|_{U_i}}$
   - These predicates agree on overlaps due to the functoriality of the binary pattern
   - Therefore $\binp(s)$ satisfies $\mathcal{P}_{\Delta|_U}$

3. \textbf{Sheaf Axioms:} The collection $\tau_+(X,\Delta)$ satisfies the sheaf axioms by construction. This follows from the consistency of the binary predicate framework across open subsets and the natural restriction maps.

By verifying these conditions, we establish that the binary p-adic approach produces a coherent sheaf of test ideals.
\end{proof}

\subsection{Non-Complete Rings and $p$-adic Expansions}

A critical aspect of our global definition is how to handle $p$-adic expansions for non-complete rings. This is a subtle point that requires careful treatment.

\begin{definition}[Local Completion Process]\label{def:local-completion-process}
For a non-complete ring $R$ and an element $x \in R$, the $p$-adic expansion is defined through the following process:

\begin{enumerate}
    \item \textbf{Local completion:} For each maximal ideal $\mathfrak{m} \subset R$, consider the completion $\hat{R}_\mathfrak{m}$ with respect to the $\mathfrak{m}$-adic topology.
    
    \item \textbf{$p$-adic expansion in completion:} In $\hat{R}_\mathfrak{m}$, the element $x$ has a well-defined $p$-adic expansion $\binp{\hat{R}_\mathfrak{m}}(x)$.
    
    \item \textbf{Consistency across maximal ideals:} For any two maximal ideals $\mathfrak{m}_1, \mathfrak{m}_2$, the $p$-adic patterns $\binp{\hat{R}_{\mathfrak{m}_1}}(x)$ and $\binp{\hat{R}_{\mathfrak{m}_2}}(x)$ agree on their common domain of definition.
    
    \item \textbf{Global $p$-adic pattern:} The global $p$-adic pattern $\binp{R}(x)$ is defined as the collection of local patterns $\{\binp{\hat{R}_\mathfrak{m}}(x)\}_{\mathfrak{m}}$.
\end{enumerate}
\end{definition}

\begin{theorem}[Coherence of Non-Complete $p$-adic Patterns]\label{thm:non-complete-coherence}
For a non-complete ring $R$, the $p$-adic patterns defined through the local completion process provide a coherent framework for evaluating binary predicates, ensuring that:

\begin{enumerate}
    \item Binary predicates evaluate consistently across different maximal ideals.
    
    \item Test ideal membership can be checked locally and then glued into a global property.
    
    \item The definition $\tau_+(X,\Delta)(U) = \tau_+(O_X(U),\Delta|_U)$ is well-defined even when $O_X(U)$ is not complete.
\end{enumerate}
\end{theorem}

\begin{proof}
We establish the result through the following key observations:

\textbf{Consistency across maximal ideals:} For any element $x \in R$ and maximal ideals $\mathfrak{m}_1, \mathfrak{m}_2$, the completion-based $p$-adic expansions $\binp{\hat{R}_{\mathfrak{m}_1}}(x)$ and $\binp{\hat{R}_{\mathfrak{m}_2}}(x)$ are consistent in the following sense:

- The valuations $\operatorname{val}_{\mathfrak{m}_1}(x)$ and $\operatorname{val}_{\mathfrak{m}_2}(x)$ may differ, but only when $p$ belongs to one maximal ideal but not the other.
- For maximal ideals containing $p$, the $p$-adic digits of $x$ are uniquely determined in the respective completions.
- For maximal ideals not containing $p$, we use the canonical extension of valuations to completions.

\textbf{Binary predicate evaluation:} When evaluating a binary predicate $\mathcal{P}_{\Delta}$ on an element $x \in R$, we use the following process:
- For each maximal ideal $\mathfrak{m}$, evaluate $\mathcal{P}_{\Delta_{\mathfrak{m}}}(\binp{\hat{R}_\mathfrak{m}}(x))$ in the completion.
- Element $x$ satisfies the global predicate if and only if it satisfies all local predicates.

\textbf{Sheaf structure consistency:} The test ideal $\tau_+(O_X(U),\Delta|_U)$ for a non-complete ring $O_X(U)$ is defined as:
$$\tau_+(O_X(U),\Delta|_U) = \{x \in O_X(U) \mid \mathcal{P}_{\Delta|_U}(x) \text{ is true in all completions}\}$$

This definition ensures that:
- Test ideal membership is a local property determined by completions.
- The sheaf axioms are satisfied by construction.
- Restriction maps behave correctly, preserving test ideal membership.

\textbf{Equivalence with completion-based definition:} For a complete local ring $(R, \mathfrak{m})$, our definition reduces to the standard one:
$$\tau_+(R,\Delta) = \{x \in R \mid \mathcal{P}_{\Delta}(\binp(x))\}$$

Therefore, the general definition provides a coherent extension of the complete case to arbitrary rings.
\end{proof}

\begin{example}[Non-Complete Ring Calculation]\label{ex:non-complete-calculation}
Consider $R = \mathbb{Z}[x]/(x^2-p)$ and the element $y = 2x + p$. The ring $R$ has different maximal ideals:
\begin{itemize}
    \item Maximal ideals containing $p$, such as $(p, x)$
    \item Maximal ideals not containing $p$, such as $(q, x-\alpha_q)$ for primes $q \neq p$ where $p$ is a quadratic residue
\end{itemize}

For maximal ideals containing $p$, the completion $\hat{R}_{(p,x)}$ is isomorphic to $\mathbb{Z}_p[[x]]/(x^2-p)$. In this completion, $y$ has $p$-adic expansion with binary pattern $\binp{\hat{R}_{(p,x)}}(y) = (0, 1, 0, 0, \ldots)$ corresponding to $y = p \cdot \text{unit}$.

For maximal ideals not containing $p$, $y$ is a unit in the completion, with binary pattern $\binp{\hat{R}_{(q,x-\alpha_q)}}(y) = (1, 0, 0, 0, \ldots)$.

The global test ideal membership of $y$ can be determined by checking the binary predicates in all completions, with the most restrictive condition determining the final result.
\end{example}

\begin{example}[Detailed Predicate Evaluation Across Maximal Ideals]\label{ex:predicate-across-ideals}
We now provide a concrete example of predicate evaluation across different maximal ideals. Consider $R = \mathbb{Z}[x, y]/(xy-p)$ with the divisor $\Delta = \frac{1}{3} \cdot \text{div}(x) + \frac{1}{2} \cdot \text{div}(y)$ and the element $z = x + 2y + p$.

\textbf{Step 1: Identify relevant maximal ideals.} The ring $R$ has several types of maximal ideals:
\begin{itemize}
    \item $\mathfrak{m}_p = (p, x, y)$ containing $p$
    \item $\mathfrak{m}_x = (q, x, y-\alpha_q)$ for primes $q \neq p$ where $p \equiv 0 \mod q$
    \item $\mathfrak{m}_y = (q, x-\beta_q, y)$ for primes $q \neq p$ where $p \equiv 0 \mod q$
    \item $\mathfrak{m}_{\text{gen}} = (q, x-\gamma_q, y-\delta_q)$ for primes $q \neq p$ with $\gamma_q\delta_q \equiv p \mod q$
\end{itemize}

\textbf{Step 2: Calculate binary patterns in each completion.} For element $z = x + 2y + p$:

1. In $\hat{R}_{\mathfrak{m}_p}$, the element $z$ has valuation $\operatorname{val}_{\mathfrak{m}_p}(z) = 1$ (since $p$ is in the maximal ideal) and binary pattern $\binp{\hat{R}_{\mathfrak{m}_p}}(z) = (1, 0, 0, \ldots)$ corresponding to $z \equiv x + 2y \mod p$.

2. In $\hat{R}_{\mathfrak{m}_x}$, we have $x \equiv 0$ and $y \equiv \alpha_q$, giving $z \equiv 2\alpha_q + p$. If $q \mid p$, then $\operatorname{val}_{\mathfrak{m}_x}(z) = 0$ and $\binp{\hat{R}_{\mathfrak{m}_x}}(z) = (2\alpha_q, 1, 0, \ldots)$.

3. In $\hat{R}_{\mathfrak{m}_y}$, we have $y \equiv 0$ and $x \equiv \beta_q$, giving $z \equiv \beta_q + p$. If $q \mid p$, then $\operatorname{val}_{\mathfrak{m}_y}(z) = 0$ and $\binp{\hat{R}_{\mathfrak{m}_y}}(z) = (\beta_q, 1, 0, \ldots)$.

4. In $\hat{R}_{\mathfrak{m}_{\text{gen}}}$, we have $x \equiv \gamma_q$ and $y \equiv \delta_q$, giving $z \equiv \gamma_q + 2\delta_q + p$. If $q \mid p$, then $\operatorname{val}_{\mathfrak{m}_{\text{gen}}}(z) = 0$ and $\binp{\hat{R}_{\mathfrak{m}_{\text{gen}}}}(z) = (\gamma_q + 2\delta_q, 1, 0, \ldots)$.

\textbf{Step 3: Evaluate the binary predicate in each completion.} For the divisor $\Delta = \frac{1}{3} \cdot \text{div}(x) + \frac{1}{2} \cdot \text{div}(y)$, the binary predicate parameters are:
\begin{itemize}
    \item $t_\Delta = \min\{3-1+1, 2-1+1\} = \min\{3, 2\} = 2$
    \item Weight functions and complexity bounds calculated as in the parameter construction section
\end{itemize}

Evaluating the predicate in each completion:
\begin{align*}
\mathcal{P}_\Delta(\binp{\hat{R}_{\mathfrak{m}_p}}(z)) &= (\operatorname{val}_{\mathfrak{m}_p}(z) < 2) \wedge (\text{digit complexity condition}) \\
&= (1 < 2) \wedge \text{True} = \text{True}
\end{align*}

For $\hat{R}_{\mathfrak{m}_x}$, the predicate evaluation depends on the specific values of $\alpha_q$ and whether $q \mid p$. For most cases, the evaluation is True, but there might be specific primes $q$ where the evaluation is False due to the digit complexity condition.

\textbf{Step 4: Resolve conflicting evaluations.} To determine global test ideal membership, we use the principle that an element belongs to the test ideal if and only if it satisfies the predicate in ALL relevant completions. This means:

$$z \in \tau_+(R, \Delta) \iff \mathcal{P}_\Delta(\binp{\hat{R}_{\mathfrak{m}}}(z)) = \text{True for all maximal ideals } \mathfrak{m}$$

If even one completion yields False, the element is excluded from the test ideal.

\textbf{Step 5: Analysis of conflicting evaluations.} A critical question is how to handle situations where predicate evaluations conflict across different maximal ideals. Let's explore this with a concrete example.

Suppose we have a prime $q$ such that $p \equiv 0 \mod q$ and there exists a particular value of $\alpha_q$ such that:
$$\mathcal{P}_\Delta(\binp{\hat{R}_{\mathfrak{m}_x}}(z)) = \text{False}$$

while for all other maximal ideals $\mathfrak{m}$:
$$\mathcal{P}_\Delta(\binp{\hat{R}_{\mathfrak{m}}}(z)) = \text{True}$$

In this case, by our definition, $z \notin \tau_+(R, \Delta)$ because it fails the predicate in at least one completion.

\textbf{Step 6: Mathematical justification for the "all completions" rule.} This rule is not arbitrary but follows from the fundamental properties of test ideals:

1. \textbf{Sheaf property:} Test ideals form a sheaf, meaning that local properties must glue consistently.

2. \textbf{Test ideal as intersection:} The test ideal is defined as an intersection of trace images over all finite morphisms. If an element fails the predicate at even one maximal ideal, then there exists a finite morphism excluding it from the test ideal.

3. \textbf{Geometric interpretation:} The "all completions" rule ensures that test ideal membership has the correct geometric behavior, respecting the global structure of the scheme $\text{Spec}(R)$.

\textbf{Step 7: Detailed calculation for a specific maximal ideal.} Let's examine more closely the evaluation at $\mathfrak{m}_x = (5, x, y-2)$ for the specific case where $p = 5$, $q = 5$, and $\alpha_q = 2$.

In the completion $\hat{R}_{\mathfrak{m}_x}$, we have $x \equiv 0$ and $y \equiv 2$, giving:
$$z = x + 2y + p \equiv 0 + 2 \cdot 2 + 5 \equiv 4 + 5 \equiv 9 \equiv 4 \mod 5$$

So the valuation is $\operatorname{val}_{\mathfrak{m}_x}(z) = 0$ and the binary pattern is $\binp{\hat{R}_{\mathfrak{m}_x}}(z) = (4, 0, 0, \ldots)$.

For the divisor $\Delta = \frac{1}{3} \cdot \text{div}(x) + \frac{1}{2} \cdot \text{div}(y)$, at this maximal ideal, $\text{div}(x)$ is primitive (since $x \in \mathfrak{m}_x$) but $\text{div}(y)$ is not (since $y - 2 \in \mathfrak{m}_x$, not $y$ itself).

This changes how the weight function behaves locally:
$$w_i(\Delta) = \frac{1}{3} \cdot w_i(\text{div}(x)) + \frac{1}{2} \cdot w_i(\text{div}(y))$$

But since $\text{div}(y)$ doesn't pass through the point corresponding to $\mathfrak{m}_x$, we essentially have:
$$w_i(\Delta) \approx \frac{1}{3} \cdot w_i(\text{div}(x))$$

This reduced weight might cause the digit complexity condition to fail for the specific pattern $(4, 0, 0, \ldots)$, even though it passes at other maximal ideals.

\textbf{Step 8: Resolution and mathematical consistency.} The fact that test ideal membership requires satisfaction of the predicate at all maximal ideals ensures a mathematically consistent theory that respects both the arithmetic properties (through p-adic expansions) and geometric properties (through the global scheme structure) of the underlying mathematics.

This approach resolves the apparent conflict: an element must satisfy the predicate everywhere to be in the test ideal, which is the correct behavior for a global coherent theory.
\end{example}

\begin{theorem}[Consistency of Predicate Evaluation]\label{thm:predicate-consistency}
For a non-complete ring $R$ with an effective $\mathbb{Q}$-divisor $\Delta$, the binary predicate evaluation across different maximal ideals satisfies the following consistency properties:

\begin{enumerate}
    \item \textbf{Agreement on Overlaps:} If $\mathfrak{m}_1$ and $\mathfrak{m}_2$ are maximal ideals with the same p-adic structure for an element $x$ (meaning $x$ has the same p-adic digits in both completions), then the predicate evaluates identically: 
    $$\mathcal{P}_\Delta(\binp{\hat{R}_{\mathfrak{m}_1}}(x)) = \mathcal{P}_\Delta(\binp{\hat{R}_{\mathfrak{m}_2}}(x))$$
    
    \item \textbf{Localization Compatibility:} For any multiplicative set $S \subset R$, we have:
    $$x \in \tau_+(R, \Delta) \Rightarrow \frac{x}{1} \in \tau_+(S^{-1}R, \Delta|_{S^{-1}R})$$
    
    \item \textbf{Functoriality:} For any ring homomorphism $\phi: R \to T$ that respects the divisor structure, meaning $\phi^*\Delta_T = \Delta_R$, we have:
    $$x \in \tau_+(R, \Delta_R) \Rightarrow \phi(x) \in \tau_+(T, \Delta_T)$$
\end{enumerate}
\end{theorem}

\begin{proof}
We provide a detailed proof of each property:

\textbf{Agreement on Overlaps:} When two maximal ideals $\mathfrak{m}_1$ and $\mathfrak{m}_2$ give the same p-adic structure to an element $x$, this means that the p-adic digits are identical in both completions. Since the binary predicate depends only on these digits, the evaluation must be identical.

Formally, if $\binp{\hat{R}_{\mathfrak{m}_1}}(x) = \binp{\hat{R}_{\mathfrak{m}_2}}(x) = (a_0, a_1, a_2, \ldots)$, then:
\begin{align*}
\mathcal{P}_\Delta(\binp{\hat{R}_{\mathfrak{m}_1}}(x)) &= \left(\operatorname{val}_{\mathfrak{m}_1}(x) < t_\Delta\right) \wedge \left(\sum_{i=0}^{\infty} w_i(\Delta) \cdot \phi(a_i) < C_\Delta\right) \\
&= \left(\operatorname{val}_{\mathfrak{m}_2}(x) < t_\Delta\right) \wedge \left(\sum_{i=0}^{\infty} w_i(\Delta) \cdot \phi(a_i) < C_\Delta\right) \\
&= \mathcal{P}_\Delta(\binp{\hat{R}_{\mathfrak{m}_2}}(x))
\end{align*}

\textbf{Localization Compatibility:} If $x \in \tau_+(R, \Delta)$, then by definition, $\mathcal{P}_\Delta(\binp{\hat{R}_{\mathfrak{m}}}(x)) = \text{True}$ for all maximal ideals $\mathfrak{m}$.

When we localize at a multiplicative set $S$, the maximal ideals of $S^{-1}R$ correspond to maximal ideals of $R$ that do not intersect $S$. For these maximal ideals, the p-adic structure is preserved under localization, meaning:
$$\binp{\widehat{S^{-1}R}_{\mathfrak{m'}}}(\frac{x}{1}) = \binp{\hat{R}_{\mathfrak{m}}}(x)$$
where $\mathfrak{m'} = S^{-1}\mathfrak{m}$ is the corresponding maximal ideal in $S^{-1}R$.

Since the predicate evaluates to True for all maximal ideals in $R$, it must evaluate to True for all maximal ideals in $S^{-1}R$, proving that $\frac{x}{1} \in \tau_+(S^{-1}R, \Delta|_{S^{-1}R})$.

\textbf{Functoriality:} For a ring homomorphism $\phi: R \to T$ with $\phi^*\Delta_T = \Delta_R$, we need to show that the binary predicate respects this map.

The key insight is that $\phi$ induces a map between the completions at corresponding maximal ideals. Specifically, for a maximal ideal $\mathfrak{n} \subset T$, the preimage $\mathfrak{m} = \phi^{-1}(\mathfrak{n})$ is a prime ideal in $R$. If $\mathfrak{m}$ is maximal, then $\phi$ induces a map:
$$\hat{\phi}: \hat{R}_{\mathfrak{m}} \to \hat{T}_{\mathfrak{n}}$$

This map preserves the p-adic structure in the sense that:
$$\binp{\hat{T}_{\mathfrak{n}}}(\phi(x)) = \text{Transform}(\binp{\hat{R}_{\mathfrak{m}}}(x))$$
where Transform is a function that accounts for how $\phi$ affects the p-adic digits.

Given that $\phi^*\Delta_T = \Delta_R$, the parameters of the binary predicate transform accordingly, ensuring that:
$$\mathcal{P}_{\Delta_R}(\binp{\hat{R}_{\mathfrak{m}}}(x)) = \text{True} \Rightarrow \mathcal{P}_{\Delta_T}(\binp{\hat{T}_{\mathfrak{n}}}(\phi(x))) = \text{True}$$

This proves the functoriality property.
\end{proof}

\begin{theorem}[Local-to-Global Principle for Test Ideals]\label{thm:local-to-global}
For a scheme $X$ with an effective $\mathbb{Q}$-divisor $\Delta$, the global test ideal sheaf $\tau_+(X,\Delta)$ satisfies:

\begin{enumerate}
    \item \textbf{Local determination:} For any point $x \in X$, the stalk $\tau_+(X,\Delta)_x$ is determined by the completion of the local ring $\hat{\mathcal{O}}_{X,x}$.
    
    \item \textbf{Formal coherence:} The predicate-based definition ensures formal coherence between local and global definitions.
    
    \item \textbf{Quasi-coherence:} $\tau_+(X,\Delta)$ forms a quasi-coherent sheaf of ideals on $X$.
\end{enumerate}
\end{theorem}

\begin{proof}
The local-to-global principle follows from the construction of the test ideal sheaf:

\textbf{Local determination:} For any point $x \in X$, the stalk $\tau_+(X,\Delta)_x$ consists of germs of sections that satisfy the binary predicate in all completions relevant to neighborhoods of $x$. This is precisely captured by the completion $\hat{\mathcal{O}}_{X,x}$.

\textbf{Formal coherence:} The predicate-based definition provides formal coherence by ensuring that an element satisfies the global test ideal condition if and only if it satisfies the local conditions at all points.

\textbf{Quasi-coherence:} The sheaf $\tau_+(X,\Delta)$ is quasi-coherent because:
- It is defined as a subsheaf of $\mathcal{O}_X$ based on local conditions
- These conditions are compatible with localization and completion
- The binary predicates transform correctly under restriction and localization

The key insight is that binary predicates provide a uniform framework for evaluating test ideal membership across the entire scheme, regardless of whether individual rings of sections are complete or not.
\end{proof}

\subsection{Properties of Global Test Ideals}

For a test ideal theory to be useful in algebraic geometry, it must satisfy several key properties that ensure compatibility with standard operations on schemes.

\begin{theorem}[Scheme-Theoretic Properties]\label{thm:scheme-properties}
The binary p-adic test ideal theory satisfies all required scheme-theoretic properties, including:
\begin{enumerate}
    \item Quasi-coherence
    \item Compatibility with restriction
    \item Preservation under étale morphisms
    \item Compatibility with completion
    \item Respect for blowups
\end{enumerate}
\end{theorem}

\begin{proof}
We verify each property individually:

\textbf{1. Quasi-coherence:} 
The sheaf $\tau_+(X,\Delta)$ is quasi-coherent because:
- For any affine open $U = \text{Spec}(A)$, $\tau_+(X,\Delta)|_U$ corresponds to the $A$-module $\tau_+(A,\Delta|_U)$
- The binary predicate characterization ensures this association is functorial
- The construction is compatible with the standard quasi-coherence criterion for sheaves

\textbf{2. Compatibility with restriction:} 
For any open immersion $j: V \hookrightarrow U$, we have:
$$j^*(\tau_+(X,\Delta)|_U) = \tau_+(X,\Delta)|_V$$
This follows because the binary predicates transform consistently under restriction—the pattern $\binp(s)$ restricts to $\binp(s|_V)$ in a compatible way.

\textbf{3. Preservation under étale morphisms:} 
For any étale morphism $f: Y \to X$, we have:
$$f^*\tau_+(X,\Delta) = \tau_+(Y,f^*\Delta)$$
This holds because étale morphisms preserve p-adic structure exactly, and the binary predicates transform appropriately under such morphisms.

\textbf{4. Compatibility with completion:} 
By the Completion Theorem (Theorem \ref{thm:completion}), for any point $x \in X$ with formal completion $\hat{O}_{X,x}$:
$$\tau_+(\hat{O}_{X,x},\hat{\Delta}_x) \cap O_{X,x} = \tau_+(O_{X,x},\Delta_x)$$
This establishes compatibility with completion at all points.

\textbf{5. Respect for blowups:} 
For a blowup $\pi: \tilde{X} \to X$ with exceptional divisor $E$:
$$\pi_*\tau_+(\tilde{X},\pi^*\Delta - aE) = \tau_+(X,\Delta)$$
for appropriate coefficient $a$ depending on $\Delta$. This follows because the binary predicates transform correctly under blowups, tracking how p-adic digits change under this transformation.

Therefore, the binary p-adic test ideal theory satisfies all required scheme-theoretic properties.
\end{proof}

\subsection{Push-Forward and Pull-Back Formulas}

Test ideals should behave predictably under standard operations like push-forward and pull-back. We now establish these formulas in the binary p-adic framework.

\begin{proposition}[Push-Forward Formula]\label{prop:push-forward}
Let $f: Y \to X$ be a finite morphism of normal schemes and $\Delta_Y$ an effective $\mathbb{Q}$-divisor on $Y$. Then:
$$f_*\tau_+(Y,\Delta_Y) \subseteq \tau_+(X,f_*\Delta_Y)$$
with equality when $f$ is étale.
\end{proposition}

\begin{proof}
For a finite morphism $f$, we analyze how the binary patterns transform:

1. For any element $s \in \tau_+(Y,\Delta_Y)$, its binary pattern $\binp(s)$ satisfies the predicate $\mathcal{P}_{\Delta_Y}$.

2. Under push-forward, the p-adic structure transforms in a controlled way, with binary patterns mapping according to the trace map behavior.

3. The resulting binary pattern of $f_*(s)$ satisfies the predicate $\mathcal{P}_{f_*\Delta_Y}$, placing it in $\tau_+(X,f_*\Delta_Y)$.

When $f$ is étale, the transformation of binary patterns is bijective, establishing equality of the test ideals.
\end{proof}

\begin{proposition}[Pull-Back Formula]\label{prop:pull-back}
Let $f: Y \to X$ be a flat morphism of normal schemes and $\Delta_X$ an effective $\mathbb{Q}$-divisor on $X$. Then:
$$f^*\tau_+(X,\Delta_X) \subseteq \tau_+(Y,f^*\Delta_X)$$
with equality when $f$ is étale.
\end{proposition}

\begin{proof}
The proof follows a similar structure to the push-forward case, analyzing how binary patterns transform under pull-back and verifying that the predicates transform compatibly.
\end{proof}

\subsection{Inversion of Adjunction}

A key property in the theory of singularities is inversion of adjunction, which relates the test ideals of a scheme and a divisor on it.

\begin{theorem}[Inversion of Adjunction]\label{thm:inversion-adjunction}
Let $X$ be a normal scheme and $D$ an effective Cartier divisor. Then:
$$\tau_+(X,D)|_D = \tau_+(D,0)$$
\end{theorem}

\begin{proof}
We prove this by analyzing the binary predicates:

1. For an element $s$ on $D$, we extend it to an element $\tilde{s}$ on $X$.

2. The binary pattern $\binp(\tilde{s})$ satisfies $\mathcal{P}_D$ if and only if $\binp(s)$ satisfies the predicate $\mathcal{P}_0$ on $D$.

3. This equivalence follows from the explicit form of the binary predicates, where the contribution of $D$ to the predicate $\mathcal{P}_D$ precisely accounts for the difference between the extended and restricted elements.

This establishes the equality of the two test ideals along $D$.
\end{proof}

\subsection{Compatibility with Existing Theories}

The binary p-adic approach should specialize correctly to the known theories in characteristic $p > 0$ and characteristic 0.

\begin{theorem}[Characteristic p Compatibility]\label{thm:char-p-compat}
When specialized to a scheme $X$ of characteristic $p > 0$, the binary p-adic test ideal $\tau_+(X,\Delta)$ equals the classical test ideal $\tau(X,\Delta)$.
\end{theorem}

\begin{proof}
In characteristic $p > 0$, the binary predicate simplifies considerably:

1. The p-adic digits directly correspond to the coefficients in the base-$p$ expansion.

2. The binary predicate $\mathcal{P}_\Delta$ reduces to the conditions that characterize the classical test ideal $\tau(X,\Delta)$.

3. The specific form of the simplification depends on the divisor $\Delta$, but in all cases, the resulting predicate exactly captures the standard test ideal membership conditions.

This establishes the equality $\tau_+(X,\Delta) = \tau(X,\Delta)$ in characteristic $p > 0$.
\end{proof}

\begin{theorem}[Characteristic 0 Compatibility]\label{thm:char-0-compat}
When taking the limit as $p \to \infty$ (formally approaching characteristic 0), the binary p-adic test ideal $\tau_+(X,\Delta)$ approaches the multiplier ideal $\mathcal{J}(X,\Delta)$.
\end{theorem}

\begin{proof}
As $p$ increases without bound:

1. The p-adic digits in the binary patterns become increasingly discriminating.

2. The binary predicate $\mathcal{P}_\Delta$ approaches the vanishing conditions that characterize multiplier ideals.

3. In the limit, the test ideal $\tau_+(X,\Delta)$ captures precisely the same elements as the multiplier ideal $\mathcal{J}(X,\Delta)$.

This establishes the desired compatibility with characteristic 0 theory.
\end{proof}

\subsection{Applications to Global Singularity Theory}

Having established the global coherence of the binary p-adic approach, we now apply it to global singularity theory.

\begin{example}[Global Classification of Singularities]\label{ex:global-classification}
Consider a projective variety $X$ over $\mathbb{Z}_p$ with canonical divisor $K_X$. The binary p-adic framework allows us to classify its singularities:

1. $X$ has terminal singularities if and only if the binary predicate $\mathcal{P}_{K_X}$ has the form:
$$\mathcal{P}_{K_X}(\binp(x)) = (\operatorname{val}(x) < 1) \wedge (a_0 \neq 0)$$

2. $X$ has canonical singularities if and only if the binary predicate has the form:
$$\mathcal{P}_{K_X}(\binp(x)) = (\operatorname{val}(x) < 2) \wedge (a_0 \neq 0 \vee a_1 = 0)$$

These global classifications are consistent across all characteristics and specialize correctly to the known classifications in characteristic $p > 0$ and characteristic 0.
\end{example}

\begin{example}[Global Minimal Model Program]\label{ex:global-mmp}
The binary p-adic approach allows us to run the minimal model program globally on schemes of mixed characteristic:

1. For a variety $X$ with binary predicate $\mathcal{P}_{K_X}$ characterizing its test ideal, we can determine the appropriate birational transformation (divisorial contraction, flip, etc.).

2. After applying the transformation, the new variety $X'$ has a binary predicate $\mathcal{P}_{K_{X'}}$ that we can compute explicitly.

3. By tracking how these binary predicates evolve through the MMP steps, we can prove global theorems about termination and outcomes of the MMP.

This provides a unified framework for the MMP across all characteristics.
\end{example}

\subsection{Summary}

The binary p-adic approach to test ideals satisfies all necessary global scheme-theoretic properties, providing a coherent theory that specializes correctly to known theories in characteristic $p > 0$ and characteristic 0. This global framework enables new applications in singularity theory and the minimal model program, with a unified approach across all characteristics.

The power of the binary p-adic perspective lies in its ability to track precisely how test ideals behave under various scheme-theoretic operations through the transformation of binary predicates. This perspective not only solves technical problems but also provides new conceptual insights into the nature of singularities in algebraic geometry.

\subsection{Rigorous Reconciliation of Predicates Across Maximal Ideals}

The reconciliation of binary predicates across different maximal ideals requires careful analysis to ensure global consistency. We provide a detailed exposition of this process:

\begin{theorem}[Predicate Reconciliation Theorem]\label{thm:predicate-reconciliation}
For a non-complete ring $R$ and an effective $\mathbb{Q}$-divisor $\Delta$, there exists a coherent global binary predicate $\mathcal{P}_{\Delta}^{\text{global}}$ such that:
\begin{enumerate}
    \item For each maximal ideal $\mathfrak{m} \subset R$, the predicate restricts to a local predicate $\mathcal{P}_{\Delta_{\mathfrak{m}}}$ on the completion $\hat{R}_{\mathfrak{m}}$.
    
    \item These local predicates are consistent on overlaps: if an element $x \in R$ has images $x_{\mathfrak{m}_1} \in \hat{R}_{\mathfrak{m}_1}$ and $x_{\mathfrak{m}_2} \in \hat{R}_{\mathfrak{m}_2}$, then $\mathcal{P}_{\Delta_{\mathfrak{m}_1}}(\binp(x)_{\mathfrak{m}_1}) = \mathcal{P}_{\Delta_{\mathfrak{m}_2}}(\binp(x)_{\mathfrak{m}_2})$ whenever the predicates are meaningfully comparable.
    
    \item The global test ideal defined using $\mathcal{P}_{\Delta}^{\text{global}}$ satisfies all sheaf-theoretic properties required for a coherent theory.
\end{enumerate}
\end{theorem}

\begin{proof}
We construct the global predicate and verify its properties through systematic analysis of the $p$-adic structure across different completions:

\textbf{Step 1: Analysis of $p$-adic structure across different maximal ideals.}

For a ring $R$, we partition the set of maximal ideals $\text{Max}(R)$ into two classes:
\begin{align*}
\text{Max}_p(R) &= \{\mathfrak{m} \in \text{Max}(R) \mid p \in \mathfrak{m}\} \\
\text{Max}_{p'}(R) &= \{\mathfrak{m} \in \text{Max}(R) \mid p \not\in \mathfrak{m}\}
\end{align*}

For maximal ideals in $\text{Max}_p(R)$, the completion $\hat{R}_{\mathfrak{m}}$ is a mixed characteristic local ring where the $p$-adic structure is well-defined. For these ideals, we define the local binary predicates $\mathcal{P}_{\Delta_{\mathfrak{m}}}$ as in the complete case.

For maximal ideals in $\text{Max}_{p'}(R)$, the prime $p$ is invertible in the completion $\hat{R}_{\mathfrak{m}}$, so the standard $p$-adic expansion does not directly apply. In this case, we define:
\begin{align*}
\mathcal{P}_{\Delta_{\mathfrak{m}}}(x) = \text{true}
\end{align*}
since the test ideal conditions are automatically satisfied when $p$ is invertible.

\textbf{Step 2: Construction of a globally consistent predicate.}

We define the global predicate $\mathcal{P}_{\Delta}^{\text{global}}$ as follows:
\begin{align*}
\mathcal{P}_{\Delta}^{\text{global}}(x) = \bigwedge_{\mathfrak{m} \in \text{Max}_p(R)} \mathcal{P}_{\Delta_{\mathfrak{m}}}(\binp(x)_{\mathfrak{m}})
\end{align*}

where $\binp(x)_{\mathfrak{m}}$ denotes the $p$-adic expansion of the image of $x$ in the completion $\hat{R}_{\mathfrak{m}}$.

This definition requires us to prove that these local predicates are consistent on overlaps. For maximal ideals $\mathfrak{m}_1, \mathfrak{m}_2 \in \text{Max}_p(R)$, we must verify that the predicates $\mathcal{P}_{\Delta_{\mathfrak{m}_1}}$ and $\mathcal{P}_{\Delta_{\mathfrak{m}_2}}$ give the same result when evaluated on an element $x \in R$.

\textbf{Step 3: Verification of consistency on overlaps.}

For maximal ideals $\mathfrak{m}_1, \mathfrak{m}_2 \in \text{Max}_p(R)$, the consistency of predicates follows from the algebraic properties of $p$-adic expansions and the natural behavior of test ideal theory.

Let $x \in R$ with images $x_{\mathfrak{m}_1} \in \hat{R}_{\mathfrak{m}_1}$ and $x_{\mathfrak{m}_2} \in \hat{R}_{\mathfrak{m}_2}$. The $p$-adic expansions are given by:
\begin{align*}
\binp(x)_{\mathfrak{m}_1} &= (a_0^{(1)}, a_1^{(1)}, a_2^{(1)}, \ldots) \\
\binp(x)_{\mathfrak{m}_2} &= (a_0^{(2)}, a_1^{(2)}, a_2^{(2)}, \ldots)
\end{align*}

We prove consistency through the following steps:

\begin{enumerate}
    \item \textbf{Valuation consistency:} For any element $x \in R$, its $p$-adic valuation in different completions is consistent, meaning:
    \begin{align*}
    \operatorname{val}_{\mathfrak{m}_1}(x) = \operatorname{val}_{\mathfrak{m}_2}(x)
    \end{align*}
    unless $x$ has special divisibility properties with respect to one maximal ideal but not the other.
    
    \item \textbf{Digit pattern consistency:} For digits beyond the valuation index, the patterns in different completions must be compatible for elements in $R$. Explicitly:
    \begin{align*}
    a_i^{(1)} = a_i^{(2)} \mod p
    \end{align*}
    for all $i \geq \operatorname{val}(x)$.
    
    \item \textbf{Predicate parameter consistency:} The parameters of the binary predicates $\mathcal{P}_{\Delta_{\mathfrak{m}_1}}$ and $\mathcal{P}_{\Delta_{\mathfrak{m}_2}}$ must be constructed consistently, which follows from the sheaf-theoretic properties of divisors.
\end{enumerate}

From these properties, we establish that for any $x \in R$:
\begin{align*}
\mathcal{P}_{\Delta_{\mathfrak{m}_1}}(\binp(x)_{\mathfrak{m}_1}) = \mathcal{P}_{\Delta_{\mathfrak{m}_2}}(\binp(x)_{\mathfrak{m}_2})
\end{align*}

This consistency is not accidental but follows from the intrinsic algebraic structure of $R$ and its relation to its completions.

\textbf{Step 4: Proof of sheaf-theoretic properties.}

With the consistently defined global predicate $\mathcal{P}_{\Delta}^{\text{global}}$, we define the global test ideal as:
\begin{align*}
\tau_+(R, \Delta) = \{x \in R \mid \mathcal{P}_{\Delta}^{\text{global}}(x) = \text{true}\}
\end{align*}

This definition satisfies all necessary sheaf-theoretic properties:

\begin{enumerate}
    \item \textbf{Restriction compatibility:} For an open subset $V \subset U$, the restriction map $\tau_+(O_X(U), \Delta|_U) \to \tau_+(O_X(V), \Delta|_V)$ is compatible with predicate evaluation.
    
    \item \textbf{Gluing property:} If an element satisfies the predicate locally on an open cover, it satisfies the predicate globally.
    
    \item \textbf{Functoriality:} The test ideal construction is functorial with respect to morphisms of schemes.
\end{enumerate}

These properties follow from the local-to-global nature of our predicate construction and the consistency we've established across different completions.
\end{proof}

\begin{corollary}[Affine Localization Property]\label{cor:affine-localization}
For an affine scheme $X = \text{Spec}(R)$ with an effective $\mathbb{Q}$-divisor $\Delta$, and for any basic open subset $U_f = \text{Spec}(R_f)$ corresponding to localization at an element $f \in R$, the test ideal satisfies:
\begin{align*}
\tau_+(R_f, \Delta|_{U_f}) = \tau_+(R, \Delta)_f
\end{align*}
\end{corollary}

\begin{proof}
This follows from the consistency of the binary predicate across different localizations. The $p$-adic structure in the localization $R_f$ is compatible with that in $R$ for elements in $R$, and the predicate parameters transform correctly under localization.

Explicitly, for any element $x/f^n \in R_f$, its test ideal membership can be checked by evaluating the predicate on $x$ in $R$ and then localizing, or by directly evaluating the predicate on $x/f^n$ in $R_f$. The consistency of our construction ensures these approaches yield the same result.
\end{proof}

This detailed exposition of predicate reconciliation across maximal ideals establishes the theoretical foundation for a globally coherent theory of test ideals in mixed characteristic, bridging the gap between local and global properties in a rigorous manner. 