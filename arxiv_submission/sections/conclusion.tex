\section{Conclusion and Further Directions}\label{sec:conclusion}

In this paper, we have developed a comprehensive binary p-adic framework for test ideals in mixed characteristic. This novel approach has allowed us to resolve several fundamental open problems in the theory of singularities and provide a unifying perspective across all characteristics.

\subsection{Summary of Results}

Our main contributions include:

\begin{enumerate}
    \item A binary p-adic characterization of test ideals in mixed characteristic through a family of explicit predicates on the p-adic digits of ring elements.
    
    \item Resolution of the completion problem, proving that test ideals commute with completion in mixed characteristic.
    
    \item Proof of the subadditivity property for test ideals in mixed characteristic, showing that 
    $\tauplus(R, \Delta_1 + \Delta_2) \subseteq \tauplus(R, \Delta_1) \cdot \tauplus(R, \Delta_2)$.
    
    \item Unification of various formulations of test ideals (standard, trace-based, perfectoid, and tight closure) under a single binary p-adic framework.
    
    \item Verification of global scheme-theoretic properties necessary for a coherent theory, including compatibility with restriction, preservation under étale morphisms, and respect for blow-ups.
    
    \item A computational framework for determining test ideal membership through explicit algorithms based on p-adic digit patterns.
\end{enumerate}

Our binary p-adic approach provides a complete and rigorous theory of test ideals in mixed characteristic that satisfies all necessary properties for applications in birational geometry and the minimal model program.

\subsection{Further Directions}

While our framework resolves several foundational problems, it also opens new directions for research:

\begin{enumerate}
    \item \textbf{Algorithmic aspects:} Developing more efficient algorithms for computing test ideals based on our binary p-adic characterization. The explicit nature of our predicates suggests possibilities for optimization.
    
    \item \textbf{Arithmetic applications:} Exploring the connections between our binary p-adic framework and arithmetic geometry, particularly in understanding how test ideals interact with arithmetic structures.
    
    \item \textbf{Higher dimensional singularities:} Extending our framework to classify and understand more complex higher-dimensional singularities in mixed characteristic.
    
    \item \textbf{Birational geometry:} Developing a complete minimal model program in mixed characteristic using our test ideal theory, particularly for 3-folds and higher dimensional varieties.
    
    \item \textbf{Generalized binary predicates:} Investigating more general forms of binary predicates that could capture other algebraic invariants beyond test ideals.
    
    \item \textbf{Non-commutative extensions:} Exploring whether similar binary p-adic approaches could be developed for non-commutative rings and their singularity theory.
\end{enumerate}

\subsection{Concluding Remarks}

The binary p-adic framework presented in this paper represents a significant advancement in the understanding of singularities in mixed characteristic. By providing a unified perspective that bridges characteristic $p > 0$ and characteristic 0 theories, our approach offers both theoretical insight and practical computational tools.

We believe that this framework will serve as a foundation for further developments in birational geometry across all characteristics, helping to complete the minimal model program in mixed characteristic and deepening our understanding of singularities in algebraic geometry.

As the theory of test ideals continues to evolve, we anticipate that the binary p-adic perspective will reveal further connections between algebraic geometry, commutative algebra, and number theory, potentially leading to new insights in these interconnected fields. 