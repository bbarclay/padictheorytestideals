\section{Detailed Parameter Construction Algorithm}\label{app:parameter-construction}

In this appendix, I provide a detailed step-by-step algorithm for computing the predicate parameters $(t_\Delta, w_i(\Delta), \phi, C_\Delta)$ that were referenced in Theorem \ref{thm:predicate-parameters}. While the main text described the general approach, this appendix offers explicit computational procedures to enhance reproducibility and practical application.

\subsection{Algorithm for Computing Predicate Parameters}

\begin{algorithm}[H]
\caption{Predicate Parameter Computation Algorithm}
\label{alg:parameter-computation}
\begin{algorithmic}[1]
\Require Effective $\mathbb{Q}$-divisor $\Delta = \sum_{j=1}^{r} a_j \text{div}(f_j)$ with $a_j = \frac{n_j}{m_j}$ in lowest terms
\State \textbf{Compute valuation threshold $t_\Delta$:}
\State \quad $t_\Delta \gets \min_{1 \leq j \leq r} \{m_j - n_j + 1\}$
\State \textbf{Compute weight functions $w_i(\Delta)$ for each $i \geq 0$:}
\For{$j = 1$ to $r$}
    \State $\epsilon_j \gets \frac{1}{m_j}$ \Comment{Decay factor based on denominator}
    \For{$i = 0$ to $M_j$} \Comment{$M_j = \lceil m_j \cdot \log_p(2m_j) \rceil$ is a practical upper bound}
        \State $\psi_{i,j} \gets \frac{p^i \cdot \text{ord}_p(\partial_{p^i}(f_j))}{\text{ord}_p(f_j)}$ \Comment{Sensitivity function}
        \State $w_{i,j} \gets a_j \cdot p^{-i\epsilon_j} \cdot \psi_{i,j}$ \Comment{Component weight}
    \EndFor
\EndFor
\State $w_i(\Delta) \gets \sum_{j=1}^{r} w_{i,j}$ for each $i$
\State \textbf{Define digit complexity function $\phi$:}
\State $\phi(0) \gets 0$, $\phi(a) \gets 1$ for all $a \neq 0$
\State \textbf{Compute complexity bound $C_\Delta$:}
\For{$j = 1$ to $r$}
    \State Compute binary pattern $\text{binp}(f_j) = (b_0, b_1, \ldots, b_{N_j})$ up to index $N_j$
    \State $\theta_j(i) \gets \frac{1}{1 + p^{i/m_j}} \cdot \frac{n_j}{m_j}$ for $0 \leq i \leq N_j$ \Comment{Position-specific correction}
    \State $C_j \gets a_j \cdot \left(1 + \sum_{i=0}^{N_j} w_i(\Delta) \cdot \phi(b_i) \cdot (1 + \theta_j(i))\right)$ \Comment{Component bound}
\EndFor
\State $C_\Delta \gets \sum_{j=1}^{r} C_j$ \Comment{Global complexity bound}
\Ensure Predicate parameters $(t_\Delta, w_i(\Delta), \phi, C_\Delta)$
\end{algorithmic}
\end{algorithm}

\subsection{Computing the $p$-adic Differential Operator}

The algorithm references a $p$-adic differential operator $\partial_{p^i}$, which I now define precisely:

\begin{definition}[$p$-adic Differential Operator]
For an element $f \in R$ with $p$-adic expansion $f = \sum_{j=0}^{\infty} b_j p^j$, the differential operator $\partial_{p^i}$ is given by:
$$\partial_{p^i}(f) = \frac{\partial f}{\partial b_i} = p^i + \sum_{k > i} C_{k,i} \cdot p^k$$
where the coefficients $C_{k,i}$ account for carry effects in $p$-adic arithmetic.
\end{definition}

For practical computation, I can use the following algorithm:

\begin{algorithm}[H]
\caption{Computing $\partial_{p^i}(f)$}
\label{alg:p-adic-differential}
\begin{algorithmic}[1]
\Require Element $f \in R$, digit position $i$
\State Express $f$ as $f = \sum_{j=0}^{N} b_j p^j$ to sufficient precision $N$
\State Create perturbed element $f' = f + \epsilon \cdot p^i$ with $\epsilon$ small
\State Compute $p$-adic expansions of $f$ and $f'$
\State $\partial_{p^i}(f) \approx \frac{f' - f}{\epsilon}$
\Ensure Approximation of $\partial_{p^i}(f)$
\end{algorithmic}
\end{algorithm}

\subsection{Worked Examples}

To illustrate the parameter computation algorithm, I provide several detailed examples of increasing complexity.

\begin{example}[Simple Divisor]
For $\Delta = \frac{1}{2} \cdot \text{div}(x)$ in $R = \mathbb{Z}_p[[x,y]]$, I compute:

\begin{enumerate}
    \item \textbf{Valuation threshold:} $t_\Delta = 2 - 1 + 1 = 2$
    
    \item \textbf{Weight function:} For $f_1 = x$, I have:
    \begin{align*}
    \epsilon_1 &= \frac{1}{2} \\
    \psi_{0,1} &= \frac{p^0 \cdot \text{ord}_p(\partial_{p^0}(x))}{\text{ord}_p(x)} = 1 \\
    \psi_{i,1} &= 0 \text{ for } i > 0 \text{ (since $x$ is $p$-adically simple)} \\
    w_0(\Delta) &= \frac{1}{2} \cdot p^0 \cdot 1 = \frac{1}{2} \\
    w_i(\Delta) &= \frac{1}{2} \cdot p^{-i/2} \cdot 0 = 0 \text{ for } i > 0
    \end{align*}
    
    \item \textbf{Digit complexity:} $\phi(0) = 0$, $\phi(a) = 1$ for $a \neq 0$
    
    \item \textbf{Complexity bound:} Computing $\text{binp}(x) = (1, 0, 0, \ldots)$ and $\theta_1(i) = \frac{1}{1 + p^{i/2}} \cdot \frac{1}{2}$:
    \begin{align*}
    C_\Delta &= \frac{1}{2} \cdot \left(1 + \sum_{i=0}^{1} w_i(\Delta) \cdot \phi(\text{binp}(x)_i) \cdot (1 + \theta_1(i))\right) \\
    &= \frac{1}{2} \cdot \left(1 + \frac{1}{2} \cdot 1 \cdot (1 + \frac{1}{1 + p^0} \cdot \frac{1}{2})\right) \\
    &= \frac{1}{2} \cdot (1 + \frac{1}{2} \cdot \frac{3}{2}) = \frac{1}{2} \cdot (1 + \frac{3}{4}) = \frac{7}{8}
    \end{align*}
\end{enumerate}

This gives the binary predicate:
$$\mathcal{P}_\Delta(\text{binp}(z)) = (\text{val}(z) < 2) \wedge \left(\sum_{i=0}^{\infty} \frac{1}{2} p^{-i/2} \phi(a_i) < \frac{7}{8}\right)$$

Which simplifies to:
$$\mathcal{P}_\Delta(\text{binp}(z)) = (\text{val}(z) < 2) \wedge (a_0 \neq 0 \vee a_1 = 0)$$
\end{example}

\begin{example}[Multiple Components]
For $\Delta = \frac{1}{3} \cdot \text{div}(x) + \frac{1}{4} \cdot \text{div}(y)$ in $R = \mathbb{Z}_p[[x,y]]$, I compute:

\begin{enumerate}
    \item \textbf{Valuation threshold:} $t_\Delta = \min\{3 - 1 + 1, 4 - 1 + 1\} = \min\{3, 4\} = 3$
    
    \item \textbf{Weight function:} Computing for each component and summing:
    \begin{align*}
    w_0(\Delta) &= \frac{1}{3} \cdot 1 + \frac{1}{4} \cdot 1 = \frac{7}{12} \\
    w_i(\Delta) &= \frac{1}{3} \cdot p^{-i/3} + \frac{1}{4} \cdot p^{-i/4} \text{ for } i > 0
    \end{align*}
    
    \item \textbf{Complexity bound:} After similar calculations:
    $$C_\Delta = \frac{1}{3} \cdot C_x + \frac{1}{4} \cdot C_y \approx 1.21$$
\end{enumerate}

The resulting predicate discriminates more finely between various $p$-adic patterns.
\end{example}

\begin{example}[Singular Variety with Boundary Divisor]
For a singular variety $X = \text{Spec}(R)$ where $R = \mathbb{Z}_p[[x,y,z]]/(xy-z^2)$ with $\Delta = \frac{4}{5} \cdot \text{div}(x)$, I compute:

\begin{enumerate}
    \item \textbf{Valuation threshold:} $t_\Delta = 5 - 4 + 1 = 2$
    
    \item \textbf{Weight function:} The singularity affects how digits interact:
    \begin{align*}
    w_0(\Delta) &= \frac{4}{5} \\
    w_1(\Delta) &= \frac{4}{5} \cdot p^{-1/5} \cdot (1 + \gamma) \approx 0.73 \cdot (1 + \gamma)
    \end{align*}
    where $\gamma \approx 0.2$ accounts for the singularity's effect on digit interactions.
    
    \item \textbf{Complexity bound:} The calculation yields $C_\Delta \approx 1.35$.
\end{enumerate}

The predicate captures how the singularity affects test ideal membership.
\end{example}

These examples demonstrate how the parameter computation algorithm adapts to divisors of varying complexity and rings with different structures.

\section{Analysis of Prime Dependence}\label{app:prime-dependence}

This appendix addresses the framework's dependence on the specific prime $p$ and explores how predicates adapt when $p$ changes or in settings with multiple primes.

\subsection{Behavior Under Changing Prime}

For a fixed divisor $\Delta$, the binary predicate parameters transform systematically as $p$ changes:

\begin{proposition}[Prime Scaling Relations]
If $\mathcal{P}_\Delta^{(p)}$ denotes the binary predicate for prime $p$ and $\mathcal{P}_\Delta^{(q)}$ for prime $q$, then:
\begin{enumerate}
    \item The valuation threshold $t_\Delta$ remains invariant: $t_\Delta^{(p)} = t_\Delta^{(q)}$
    \item The weight functions scale approximately as: $w_i^{(q)}(\Delta) \approx w_i^{(p)}(\Delta) \cdot \left(\frac{p}{q}\right)^{i\epsilon}$ for some $\epsilon > 0$
    \item The complexity bound adjusts to: $C_\Delta^{(q)} \approx C_\Delta^{(p)} \cdot \left(1 + \log_p{q} \cdot \rho(\Delta)\right)$ where $\rho(\Delta)$ is a divisor-dependent factor
\end{enumerate}
\end{proposition}

\begin{proof}[Sketch]
The valuation threshold depends only on the rational coefficients of $\Delta$, not on $p$. The weight functions transform due to the relative expansionary properties of $p$-adic vs. $q$-adic digits. The complexity bound adjusts to account for digit representation changes between different prime bases.
\end{proof}

\subsection{Multiple Prime Framework}

In settings with multiple primes (e.g., global fields), I can define a composite predicate:

\begin{definition}[Multi-Prime Binary Predicate]
For a set of primes $\{p_1, p_2, \ldots, p_n\}$ and a divisor $\Delta$, the multi-prime binary predicate is:
$$\mathcal{P}_\Delta^{\text{multi}}(x) = \bigwedge_{j=1}^{n} \mathcal{P}_\Delta^{(p_j)}(\text{bin}_{p_j}(x))$$
where $\text{bin}_{p_j}(x)$ is the $p_j$-adic binary pattern of $x$.
\end{definition}

This composite predicate is consistent with the individual prime-specific predicates and provides a unified framework for settings with multiple characteristics.

\begin{example}[Multi-Prime Calculation]
Consider $\Delta = \frac{1}{2} \cdot \text{div}(x)$ in $\mathbb{Z}[x,y]$, and primes $p=2, q=3$. The predicates $\mathcal{P}_\Delta^{(2)}$ and $\mathcal{P}_\Delta^{(3)}$ differ in their weight functions but share the threshold $t_\Delta = 2$. An element $x = 10$ has different binary patterns in each system:
\begin{align*}
\text{bin}_2(10) &= (0,1,0,1) \text{ in base 2} \\
\text{bin}_3(10) &= (1,0,1) \text{ in base 3}
\end{align*}

The multi-prime predicate evaluates both representations simultaneously.
\end{example}

\subsection{Characteristic 0 Limit Analysis}

The binary $p$-adic approach exhibits a well-defined limit as $p \to \infty$:

\begin{proposition}[Characteristic 0 Convergence Rate]
As $p \to \infty$, the binary predicate $\mathcal{P}_\Delta^{(p)}$ converges to the multiplier ideal membership condition at rate $O(1/p)$.
\end{proposition}

This implies that for sufficiently large $p$, the binary predicate provides an excellent approximation to characteristic 0 behavior, with quantifiable error bounds.

\section{Computational Complexity Analysis}\label{app:computational-complexity}

This appendix provides rigorous bounds on the computational complexity of predicate evaluation, addressing the locality property referenced in the main text.

\subsection{Digit Dependency Bounds}

\begin{theorem}[Finite Digit Dependence]
For an effective $\mathbb{Q}$-divisor $\Delta = \sum_{j=1}^{r} a_j \text{div}(f_j)$ with $a_j = \frac{n_j}{m_j}$ in lowest terms, predicate evaluation depends on at most $N_\Delta$ digits, where:
$$N_\Delta = \max\left\{\left\lceil \frac{\log(C_\Delta \cdot r)}{\log(1 + \mu)} \right\rceil, \max_{1 \leq j \leq r} \{m_j\} \right\}$$
with $\mu = \min_{1 \leq j \leq r} \{\frac{1}{m_j}\}$.
\end{theorem}

\begin{proof}
The weight function $w_i(\Delta)$ decreases exponentially as $w_i(\Delta) \leq M \cdot p^{-i\mu}$ for some constant $M$ and $\mu = \min_j \{\frac{1}{m_j}\}$. Consequently, the contribution of digits beyond position $N_\Delta$ to the weighted sum becomes negligible.

Specifically, the sum of weights beyond position $N_\Delta$ is bounded by:
$$\sum_{i > N_\Delta} w_i(\Delta) \leq M \cdot \sum_{i > N_\Delta} p^{-i\mu} = M \cdot \frac{p^{-(N_\Delta+1)\mu}}{1 - p^{-\mu}}$$

For this to be less than $\frac{1}{r \cdot C_\Delta}$ (ensuring it doesn't affect predicate evaluation), I need:
$$N_\Delta \geq \frac{\log(M \cdot (1 - p^{-\mu}) \cdot r \cdot C_\Delta) + \mu}{\mu \cdot \log(p)}$$

Simplifying and taking the ceiling gives my bound.
\end{proof}

\begin{corollary}[Computational Complexity]
The computational complexity of predicate evaluation is $O(r \cdot N_\Delta)$ for a divisor with $r$ components, which is $O(r \cdot \log(r \cdot C_\Delta))$ in terms of divisor parameters.
\end{corollary}

\subsection{Practical Implementation Considerations}

For practical implementation, I recommend:

\begin{enumerate}
    \item \textbf{Preprocessing:} Compute and store the weights $w_i(\Delta)$ up to position $N_\Delta$
    \item \textbf{Lazy evaluation:} Compute the digit expansion of elements incrementally until the predicate outcome is determined
    \item \textbf{Caching:} For repeated evaluations, cache intermediate results based on common digit prefixes
\end{enumerate}

These strategies reduce the average-case complexity significantly below the worst-case bounds.

\begin{example}[Practical Digit Dependence]
For $\Delta = \frac{1}{2} \cdot \text{div}(x)$, I have $\mu = \frac{1}{2}$, $r = 1$, and $C_\Delta \approx \frac{7}{8}$. This gives:
$$N_\Delta \approx \left\lceil \frac{\log(0.875)}{\log(1.5)} \right\rceil = 1$$

Confirming that, in practice, only the first digit beyond valuation affects predicate evaluation.

For more complex divisors like $\Delta = \frac{1}{10} \cdot \text{div}(x) + \frac{1}{15} \cdot \text{div}(y) + \frac{1}{20} \cdot \text{div}(z)$, I have $\mu = \frac{1}{20}$, $r = 3$, and $C_\Delta \approx 0.45$, giving:
$$N_\Delta \approx \left\lceil \frac{\log(0.45 \cdot 3)}{\log(1.05)} \right\rceil \approx 25$$

Thus, even for divisors with small denominators, the digit dependence remains manageable.
\end{example}

\section{Comparative Analysis}\label{app:comparative-analysis}

This appendix provides a comprehensive comparison between the binary p-adic approach and alternative methods for test ideals in mixed characteristic.

\subsection{Comparison Table}

\begin{table}[h]
\begin{tabular}{|p{3cm}|p{3cm}|p{3cm}|p{3cm}|}
\hline
\textbf{Feature} & \textbf{Binary P-adic Approach} & \textbf{Ma-Schwede Plus Closure} & \textbf{Perfectoid Techniques} \\
\hline
Theoretical foundation & P-adic digit patterns & Plus closure operation & Perfectoid spaces \\
\hline
Computational aspects & Explicit binary predicates & Implicit algebraic operations & Abstract perfectoid algebra \\
\hline
Completion theorem & Proven directly & Conjectured & Partial results \\
\hline
Subadditivity & Binary factorization framework & Open question & Open question \\
\hline
Non-complete rings & Predicate reconciliation & Local methods only & Perfectoid completion \\
\hline
Characteristic 0 limit & Explicit convergence rate & Indirect connection & Natural connection \\
\hline
Global theory & Developed in Section 8 & Limited & Limited \\
\hline
Implementation complexity & Moderate & High & Very high \\
\hline
\end{tabular}
\caption{Comparison of test ideal approaches in mixed characteristic}
\label{tab:comparison}
\end{table}

\subsection{Quantitative Performance Analysis}

\begin{proposition}[Computational Efficiency]
For a divisor $\Delta$ with $r$ components and coefficients with maximum denominator $m$, the computational complexity of test ideal membership verification is:
\begin{enumerate}
    \item Binary P-adic Approach: $O(r \cdot \log(r \cdot m))$
    \item Plus Closure Methods: $O(r \cdot m \cdot \log(p))$
    \item Perfectoid Techniques: $O(r \cdot m^2 \cdot \log(p))$
\end{enumerate}
\end{proposition}

This demonstrates the efficiency advantage of the binary approach, especially for divisors with large denominators.

\subsection{Strengths and Limitations}

\textbf{Binary P-adic Approach:}
\begin{itemize}
    \item \textbf{Strengths:} Explicit predicates, efficient computation, unified framework
    \item \textbf{Limitations:} Prime-specific, requires parameter computation
\end{itemize}

\textbf{Plus Closure Methods:}
\begin{itemize}
    \item \textbf{Strengths:} Direct algebraic interpretation, connection to tight closure
    \item \textbf{Limitations:} Computational complexity, limited global theory
\end{itemize}

\textbf{Perfectoid Techniques:}
\begin{itemize}
    \item \textbf{Strengths:} Strong theoretical foundation, natural connection to characteristic 0
    \item \textbf{Limitations:} Abstract constructions, computational difficulties
\end{itemize}

The binary p-adic approach effectively balances theoretical power with computational tractability, position it favorably compared to alternatives.

\section{Non-Complete Ring Techniques}\label{app:non-complete-rings}

This appendix provides additional techniques for simplifying predicate evaluation in non-complete rings, addressing the complexity noted in Section 8.

\subsection{Efficient Reconciliation Algorithm}

\begin{algorithm}[H]
\caption{Efficient Predicate Reconciliation}
\label{alg:efficient-reconciliation}
\begin{algorithmic}[1]
\Require Element $x \in R$, divisor $\Delta$, finite set of representatives $\{\mathfrak{m}_1, \ldots, \mathfrak{m}_k\}$ of maximal ideal classes
\State Initialize $\text{result} \gets \text{true}$
\For{$i = 1$ to $k$}
    \State Compute $\text{binp}_{\mathfrak{m}_i}(x)$ in completion $\hat{R}_{\mathfrak{m}_i}$
    \State Evaluate $p_i \gets \mathcal{P}_{\Delta_{\mathfrak{m}_i}}(\text{binp}_{\mathfrak{m}_i}(x))$
    \State $\text{result} \gets \text{result} \land p_i$
    \If{$\text{result} = \text{false}$}
        \State \textbf{break} \Comment{Early termination optimization}
    \EndIf
\EndFor
\Ensure $\text{result}$ (true if $x \in \tau_+(R,\Delta)$, false otherwise)
\end{algorithmic}
\end{algorithm}

\subsection{Representative Maximal Ideals}

A key optimization is identifying a finite set of representative maximal ideals that fully determine test ideal membership:

\begin{theorem}[Finite Representatives]
For a non-complete ring $R$ with divisor $\Delta$, there exists a finite set of maximal ideals $\{\mathfrak{m}_1, \ldots, \mathfrak{m}_k\}$ such that:
$$x \in \tau_+(R,\Delta) \iff \bigwedge_{i=1}^{k} \mathcal{P}_{\Delta_{\mathfrak{m}_i}}(\text{binp}_{\mathfrak{m}_i}(x)) = \text{true}$$
The number $k$ is bounded by the number of distinct prime factors appearing in the denominators of divisor coefficients.
\end{theorem}

This dramatically reduces the number of completions that need to be checked in practice.

\begin{example}[Efficient Reconciliation]
For $R = \mathbb{Z}[x]/(x^2-p)$ with divisor $\Delta = \frac{1}{2} \cdot \text{div}(x)$, I need only check two representative maximal ideals:
\begin{enumerate}
    \item $\mathfrak{m}_1 = (p, x)$ representing ideals containing $p$
    \item $\mathfrak{m}_2 = (q, x-\alpha_q)$ representing ideals not containing $p$
\end{enumerate}

This reduces the reconciliation problem from infinitely many maximal ideals to just two representatives.
\end{example}

\section{Additional Worked Examples}\label{app:additional-examples}

This appendix provides complex examples involving higher-dimensional varieties and divisors with irrational approximations to thoroughly test the binary p-adic framework.

\subsection{Higher-Dimensional Example}

\begin{example}[Three-Dimensional Singular Variety]
Consider $X = \text{Spec}(R)$ where $R = \mathbb{Z}_p[[x,y,z,w]]/(xy-zw)$ with divisor $\Delta = \frac{1}{3} \cdot \text{div}(x) + \frac{1}{4} \cdot \text{div}(z+w) + \frac{1}{5} \cdot \text{div}(x+y+z)$.

The predicate parameters are calculated as:
\begin{align*}
t_\Delta &= \min\{3, 4, 5\} = 3 \\
w_0(\Delta) &= \frac{1}{3} + \frac{1}{4} + \frac{1}{5} = \frac{47}{60} \\
w_1(\Delta) &\approx 0.53 \\
C_\Delta &\approx 1.47
\end{align*}

The test ideal $\tau_+(R,\Delta)$ includes elements with various digital patterns. For example:
\begin{align*}
x+z &\in \tau_+(R,\Delta) & \text{(val = 0, simple pattern)} \\
p(x+y) &\in \tau_+(R,\Delta) & \text{(val = 1, specific pattern)} \\
p^2(x+yz) &\notin \tau_+(R,\Delta) & \text{(val = 2, complex pattern violating bound)} \\
p^3 &\notin \tau_+(R,\Delta) & \text{(val = 3, exceeds threshold)}
\end{align*}

The 3D structure creates interesting pattern interactions not seen in simpler examples.
\end{example}

\subsection{Irrational Approximation}

\begin{example}[Approximating Irrational Coefficients]
While the theory is defined for rational coefficients $a_j = \frac{n_j}{m_j}$, I can approximate irrational coefficients to any desired precision. For $\Delta = \sqrt{2} \cdot \text{div}(x)$, I use the rational approximation $\frac{99}{70} \approx 1.414$ to get:

\begin{align*}
t_\Delta &= 70 - 99 + 1 = -28
\end{align*}

Since $t_\Delta < 0$, all elements satisfy the valuation condition, and the predicate simplifies to the pattern condition:
\begin{align*}
\mathcal{P}_\Delta(\text{binp}(z)) &= \left(\sum_{i=0}^{\infty} w_i(\Delta) \cdot \phi(a_i) < C_\Delta\right) \\
&\approx \left(1.414 \cdot \sum_{i=0}^{\infty} p^{-i/70} \cdot \phi(a_i) < 2.35\right)
\end{align*}

This demonstrates how the framework handles approximations of irrational coefficients, with increasing precision possible by using better rational approximations.
\end{example}

\subsection{Applied Example: Log Canonical Threshold}

\begin{example}[Computing Log Canonical Threshold]
The binary p-adic framework enables direct computation of log canonical thresholds in mixed characteristic. For the ideal $I = (x^3, y^4) \subset \mathbb{Z}_p[[x,y]]$, I find:

\begin{align*}
\text{lct}(I) &= \sup\{t > 0 \mid \tau_+(R, t \cdot \text{div}(x^3) + t \cdot \text{div}(y^4)) = R\} \\
&= \sup\{t > 0 \mid t \cdot 3 < 1 \text{ and } t \cdot 4 < 1\} \\
&= \sup\{t > 0 \mid t < \frac{1}{3} \text{ and } t < \frac{1}{4}\} \\
&= \frac{1}{4}
\end{align*}

This value agrees with both characteristic 0 and characteristic p results, confirming the correctness of my approach.
\end{example}

These additional examples demonstrate the framework's power and versatility across a range of complex scenarios. 