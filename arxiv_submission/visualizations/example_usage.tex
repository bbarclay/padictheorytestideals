\documentclass{article}
\usepackage{graphicx}

% Include the visualization commands
% LaTeX commands for including the visualizations in the paper
% Include this file in the preamble with % LaTeX commands for including the visualizations in the paper
% Include this file in the preamble with % LaTeX commands for including the visualizations in the paper
% Include this file in the preamble with \input{visualizations/latex_inclusion.tex}

% Define a custom command for including visualizations with consistent styling
\newcommand{\includeVisualization}[3][1.0]{%
  \begin{figure}[htbp]
    \centering
    \includegraphics[width=#1\textwidth]{visualizations/output/#2}
    \caption{#3}
    \label{fig:#2}
  \end{figure}
}

% Examples of usage:

% p-adic digit patterns
% \includeVisualization[0.8]{padic_digits_base_2.pdf}{Binary representation of $p$-adic digits for integers from $0$ to $20$, showing the pattern of digits $a_i$ in the $p$-adic expansion.}

% Predicate evaluation
% \includeVisualization{predicate_evaluation_p3.pdf}{Visualization of predicate evaluation $\mathcal{P}_\Delta$ for test ideal membership with $p=3$, showing the relationship between valuation and weighted sum components.}

% Subadditivity property
% \includeVisualization{subadditivity.pdf}{Illustration of the subadditivity property for test ideals, showing that $\tau_+(R,\Delta_1+\Delta_2) \subseteq \tau_+(R,\Delta_1) \cdot \tau_+(R,\Delta_2)$.}

% Completion theorem
% \includeVisualization{completion_theorem.pdf}{Visualization of the completion theorem, demonstrating that test ideals commute with completion: $\tau_+(\hat{R},\hat{\Delta}) \cap R = \tau_+(R,\Delta)$.}

% Alternative formulations
% \includeVisualization{alternative_formulations.pdf}{Unification of alternative formulations of test ideals through explicit predicate modifications.}

% Computational framework
% \includeVisualization{computational_framework.pdf}{The computational framework for test ideals in mixed characteristic, showing the pipeline from divisor input to test ideal computation.}

% You can easily adjust the width parameter for each visualization as needed
% e.g., \includeVisualization[0.9]{filename.pdf}{Caption text} 

% Define a custom command for including visualizations with consistent styling
\newcommand{\includeVisualization}[3][1.0]{%
  \begin{figure}[htbp]
    \centering
    \includegraphics[width=#1\textwidth]{visualizations/output/#2}
    \caption{#3}
    \label{fig:#2}
  \end{figure}
}

% Examples of usage:

% p-adic digit patterns
% \includeVisualization[0.8]{padic_digits_base_2.pdf}{Binary representation of $p$-adic digits for integers from $0$ to $20$, showing the pattern of digits $a_i$ in the $p$-adic expansion.}

% Predicate evaluation
% \includeVisualization{predicate_evaluation_p3.pdf}{Visualization of predicate evaluation $\mathcal{P}_\Delta$ for test ideal membership with $p=3$, showing the relationship between valuation and weighted sum components.}

% Subadditivity property
% \includeVisualization{subadditivity.pdf}{Illustration of the subadditivity property for test ideals, showing that $\tau_+(R,\Delta_1+\Delta_2) \subseteq \tau_+(R,\Delta_1) \cdot \tau_+(R,\Delta_2)$.}

% Completion theorem
% \includeVisualization{completion_theorem.pdf}{Visualization of the completion theorem, demonstrating that test ideals commute with completion: $\tau_+(\hat{R},\hat{\Delta}) \cap R = \tau_+(R,\Delta)$.}

% Alternative formulations
% \includeVisualization{alternative_formulations.pdf}{Unification of alternative formulations of test ideals through explicit predicate modifications.}

% Computational framework
% \includeVisualization{computational_framework.pdf}{The computational framework for test ideals in mixed characteristic, showing the pipeline from divisor input to test ideal computation.}

% You can easily adjust the width parameter for each visualization as needed
% e.g., \includeVisualization[0.9]{filename.pdf}{Caption text} 

% Define a custom command for including visualizations with consistent styling
\newcommand{\includeVisualization}[3][1.0]{%
  \begin{figure}[htbp]
    \centering
    \includegraphics[width=#1\textwidth]{visualizations/output/#2}
    \caption{#3}
    \label{fig:#2}
  \end{figure}
}

% Examples of usage:

% p-adic digit patterns
% \includeVisualization[0.8]{padic_digits_base_2.pdf}{Binary representation of $p$-adic digits for integers from $0$ to $20$, showing the pattern of digits $a_i$ in the $p$-adic expansion.}

% Predicate evaluation
% \includeVisualization{predicate_evaluation_p3.pdf}{Visualization of predicate evaluation $\mathcal{P}_\Delta$ for test ideal membership with $p=3$, showing the relationship between valuation and weighted sum components.}

% Subadditivity property
% \includeVisualization{subadditivity.pdf}{Illustration of the subadditivity property for test ideals, showing that $\tau_+(R,\Delta_1+\Delta_2) \subseteq \tau_+(R,\Delta_1) \cdot \tau_+(R,\Delta_2)$.}

% Completion theorem
% \includeVisualization{completion_theorem.pdf}{Visualization of the completion theorem, demonstrating that test ideals commute with completion: $\tau_+(\hat{R},\hat{\Delta}) \cap R = \tau_+(R,\Delta)$.}

% Alternative formulations
% \includeVisualization{alternative_formulations.pdf}{Unification of alternative formulations of test ideals through explicit predicate modifications.}

% Computational framework
% \includeVisualization{computational_framework.pdf}{The computational framework for test ideals in mixed characteristic, showing the pipeline from divisor input to test ideal computation.}

% You can easily adjust the width parameter for each visualization as needed
% e.g., \includeVisualization[0.9]{filename.pdf}{Caption text} 

\title{Example Usage of P-adic Test Ideal Visualizations}
\author{Brandon Barclay}
\date{\today}

\begin{document}

\maketitle

\section{Introduction}

This document demonstrates how to include the visualizations generated for the Binary P-adic Theory of Test Ideals paper. We use the custom \texttt{$\backslash$includeVisualization} command to easily incorporate these visualizations with proper formatting and captions.

\section{P-adic Digit Patterns}

\includeVisualization[0.8]{padic_digits_base_2.pdf}{Binary representation of $p$-adic digits for integers from $0$ to $20$, showing the pattern of digits $a_i$ in the $p$-adic expansion with $p=2$.}

The visualization above illustrates the binary ($p=2$) digit patterns. Similar visualizations are available for other prime bases ($p=3$ and $p=5$).

\section{Test Ideal Predicate Evaluation}

\includeVisualization{predicate_evaluation_p3.pdf}{Visualization of predicate evaluation $\mathcal{P}_\Delta$ for test ideal membership with $p=3$, showing the relationship between valuation and weighted sum components.}

This visualization demonstrates how the predicate $\mathcal{P}_\Delta(\text{bin}_p(x))$ is evaluated for different integers, showing both the valuation condition and the weighted sum condition.

\section{Subadditivity Property}

\includeVisualization{subadditivity.pdf}{Illustration of the subadditivity property for test ideals, showing that $\tau_+(R,\Delta_1+\Delta_2) \subseteq \tau_+(R,\Delta_1) \cdot \tau_+(R,\Delta_2)$.}

The visualization above illustrates the subadditivity property for test ideals, one of the key results of the paper.

\section{Completion Theorem}

\includeVisualization{completion_theorem.pdf}{Visualization of the completion theorem, demonstrating that test ideals commute with completion: $\tau_+(\hat{R},\hat{\Delta}) \cap R = \tau_+(R,\Delta)$.}

This visualization explains the completion theorem both from a set-theoretic perspective and through predicate invariance.

\section{Unification of Alternative Formulations}

\includeVisualization{alternative_formulations.pdf}{Unification of alternative formulations of test ideals through explicit predicate modifications.}

The above visualization shows how different approaches to test ideals can be unified through specific transformations of the predicate parameters.

\section{Computational Framework}

\includeVisualization{computational_framework.pdf}{The computational framework for test ideals in mixed characteristic, showing the pipeline from divisor input to test ideal computation.}

This visualization presents the computational pipeline for working with test ideals in mixed characteristic, highlighting key advantages of the approach.

\end{document} 